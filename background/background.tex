\chapter{Background}
\label{Chapt2}
This chapter will cover the basic background of rowing, the sports science which guides rowing training, ad how an athletes body responds to training stimulus. Next, a review of performance modelling will explore the development of human performance modelling since the introduction of the basic Bannister model in 1975 \autocite{Bannister1976}. The section will outline how training load and performance can be quantified, and explain the way these approximations are used in various performance models to date.

\section{An introduction to rowing training fundamentals}
Rowing is an Olympic sport, raced across a 2,000 metre course. It is classed as a power-endurance sport, this means training is focused on building aerobic, anerobic, and power while also developing rowing technique \autocite{Mäestu2005}. Most time is spent building endurance, next most time is spent building anerobic capacity, finally building strength and power through Strength and Condition sessions \autocite{Seiler2006}. There are many different approaches to how training is conducted and which energy systems are targeted. This section will discuss the basic training principles which guide training, the way athletes respond to different kinds of training loads, and how performance is evaulated in rowing. 

\subsection{Training Principles}
Generally when a coach builds a training plan they have a few factors they can work with: volume, the amount of mileage or total time spent training, sessions intensity, how hard the given session is meant to be, and finally the frequency, or the time spent in different intensity zones. There are various ways to measure intensity, including heart rate, blood lactate concentration, velocity at maximal oxygen uptake (VO2 max), and rate of perceived exertion (RPE) \autocite{Rosenblat2019}. Rowers tend to use heart rate zones or blood lactate concentration depending on access to the equipment to test blood lacate. Typically when a rower uses calculated aerobic zones each zone will be a percentage of \maxHR. Typically these zones are typically defined as follows:
\begin{description}
  \item[Z1] "Very Light" intensity, 50\% - 60\% of \maxHR
  \item[Z2] "Light" intensity, 60\%-70\% of \maxHR
  \item[Z3] "Moderate" intensity, 70\%-80\% of \maxHR
  \item[Z4] "Hard" intensity, 80\%-90\% of \maxHR
  \item[Z5] "Maximum" intesity, 90\%-100\% of \maxHR
\end{description}
The exact definition of these zones varies in the literature, as does the method to determine \maxHR. However, most high level athletes will have completed some kind of stress test to determine their \maxHR in order to train more effectively on their prescribed zones.
A rower who uses lactate based training zones might use the following zones:
\begin{description}
  \item[T1]  basic oxygen utilization training (UT2) [lactate~=~0-2 mmol/L]
  \item[T2]  oxygen utilization training (UT1) [lactate~=~2-3.5 mmol/L]
  \item[T3]  anaerobic threshold training (AT) [lactate~=~3.5-4.5 mmol/L]
  \item[T4]  oxygen transport training (TR) [lactate~=~4.5-6 mmol/L]
  \item[T5]  anaerobic capacity training (AN) [lactate $\geq$ 6 mmol/L] \autocite{Das2022}
\end{description}
Depending on how rigorous the testing protocol was, Heart Rate zones may be calculated for each zone, these may vary from the aerobic zones calculated from \maxHR. 

The most basic zone approximation approach uses three zones based around certain physiological thresholds, like, lacate thresholds ($\textnormal{LT}_\text{1}$ and $\textnormal{LT}_\text{2}$) and ventilatory thresholds. Cyclists may use critical power to determine these three basic zones, although this practice has not become popular in rowing training. The zones become simply, low-intensity, moderate-intensity, and high-intensity. 

There are a few different approaches for distributing intensity for endurance training. The three main methods are: polarised training, sweet spot or threshold training, and pyramidal training. This guides the final factor a coach considers when building a general training plan, frequency. For the purposes of comparing polarized training (POL), threshold training (THR), and pyramidal training (PYR), the more basic three zones of intensity will be used. The breakdown per zone for each training method is as follows:
\begin{description}
  \item[Polarised Training] Far more time spent in the low-intensity zone \autocite{Seiler2004}.

  \begin{description}
    \item[Low-Intensity] 75\%-85\% of total training volume
    \item[Medium-Intensity] 5\%-10\% of total training volume
    \item[High-Intensity] 5\%-10\% of total training volume  
  \end{description} 
  \item[Threshold Training] More time spent in the medium-intensity zone \autocite{Seiler2004}.

  \begin{description}
    \item[Low-Intensity] 45\%-55\% of total training volume
    \item[Medium-Intensity] 35\%-55\% of total training volume
    \item[High-Intensity] 15\%-20\% of total training volume  
  \end{description}
  \item[Pyramidal Training] Most time spent in low-intensity zone with progressively less time spent in higher zones \autocite{Selles2019}.

  \begin{description}
    \item[Low-Intensity] 75\%-85\% of total training volume
    \item[Medium-Intensity] 15\%-20\% of total training volume
    \item[High-Intensity] 5\%-10\% of total training volume  
  \end{description}
\end{description}

This report will not compare the effectiveness of different training distributions. Different distributions tend to be used by different sports, or depending on which energy system is being targeted. The use of polarized training is most common in rowing \autocite{Rosenblat2019}.


\subsection{Energy Systems}
An understanding of the aerobic and anerobic Systems

\subsection{Physiological Response to Training}
Adapations and that

\subsection{Further training related sections}
Discuss the need to include subsections for over reaching/training, detraining, tapering, etc.

\subsection{Heart Rate Variablity}
An explanation of HRV and how its used in training and recovery, include correlation with training load here, can include data from Churchill (2014) \autocite{Churchill2014} here.

\subsection{Performance}
How it is measured, what feeds into it.

\subsection{Summary}
Does what it says on the box.

\section{A Review of Performance Modelling}
A review of models that already exist, including history of modelling, starting with \textcite{Bannister1976}, and including \textcite{Edelmannnusser2002} and their comments on the efficacy of a linear systems model on an inherently non-linear biological behaviour. Essentially go through the steps done by Churchill (2014) \autocite{Churchill2014}.

\subsection{Quantifying Training Load (Fatigue)}
\subsubsection{RPE}
\subsection{TRIMP}


\subsection{Quantifying Performance}
erg score, otw results, telemetry

\subsection{Impulse-Response Models}
\subsubsection{Limitations to the Impulse-Response Model}

\subsection{Alternative Models}
Artifcial Neural Network (ANN) approaches. 
