\chapter{Background}
\label{Chapt2}

\section{Introduction to Sports Science}
Introduction to sports science below, including introduction to vocabulary and concepts.

\subsection{Characteristics of Rowing}
Rowing is an Olympic sport, raced in shells across a 2,000 metre course. It is classed as a power-endurance sport, requiring large blocks of training for anerobic metabolism, aerobic capacity, and muscular strength, endurance and power \cite{S2002}. \textbf{Further info about rowing, including training volume (can I reference myself?)} 

\subsection{Training Principles}
A discussion on the training approach generally

\subsection{Energy Systems}
An understanding of the aerobic and anerobic Systems

\subsection{Physiological Response to Training}
Adapations and that

\subsection{Further training related sections}
Discuss the need to include subsections for over reaching/training, detraining, tapering, etc.

\subsection{Heart Rate Variablity}
An explanation of HRV and how its used in training, include correlation with training load here, can include data from Churchill (2014) \cite{Churchill2014} here.

\subsection{Performance}
How it is measured, what feeds into it.

\subsection{Summary}
Does what it says on the box.

\section{A Review of Systems Modelling}
A review of models that already exist, including history of modelling, starting with Bannister et al. \cite{Calvert1976}, and including Edelmann-nusser et al. \cite{Edelmannnusser2002} and their comments on the efficacy of a linear systems model on an inherently non-linear biological behaviour. Essentially go through the steps done by Churchill (2014) \cite{Churchill2014}.

\subsection{Impulse-Response Models}
\subsubsection{Limitations to the Impulse-Response Model}

\subsection{Alternative Models}
Artifcial Neural Network (ANN) approaches. 

\subsection{Quantifying Training Load}
\subsection{Quantifying Performance}