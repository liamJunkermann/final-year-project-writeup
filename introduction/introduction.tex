\chapter{\label{ch:intro}Introduction}
This project explores the use of Machine Learning to model and predict rowing training outcomes and performances. Developing models for human performance is a complex and challenging task, with many factors influencing the outcome of a training session or cycle, and variety of factors can influence a performance when it matters. As early as 1975, linear models for performance were explored with reasonable success. Since then, many models have been developed to predict performance, with the most recent models using machine learning techniques to predict performance. This project aims to use some of the techniques developed to predict performance using rowing data. This chapter will introduce the general approach and motivation for this project, the expected outcomes, and a structure for the remainder of the report.

\section{Motivations}
The idea for this project originated from a personal passion for rowing and performance and an academic interest in data analysis and machine learning. As a competitive rower for the last 12 years, I have produced a huge amount of data, and for the last three years I have been tracking my training and recovery data continuously with a number of wearbles. At this point, I wear at least three heart rate monitors at all times in order to capture information for different platforms to provide me feedback on training, recovery, and readiness to train more. I have always been interested in the data I produce and how I can use it to improve my training habits to perform optimally and avoid illness and injury -- the largest deterrents to my athletic growth in recent years. 

Despite the amount of data I have been producing, I have yet to consistently analyse it to provide feedback on my training. I am motivated by the prospect of developing a more effective approach for feedback, one that capitalizes on the wealth of data I have accumulated. By doing so, I hope to not only improve my own performance but also contribute to the broader understanding of how data-driven approaches can be utilized in training prescription and personalisation.

This project was an opportunity to blend my love for rowing and sports science with my academic interest in data analysis and machine learning, ultimately driving towards tangible improvements in training and performance through the power of data analysis and machine learning.

% I love data, I love rowing, I want to use my data to row faster :)

\section{Goals}
There a number of steps to this project, with the main goal being to develop a model which can predict rowing performance. This will be done by collecting data from rowers, performing initial analysis on this data to help develop features for a model, and produce visualisations as feedback for participants. Finally, the goal was to use this data to train a model which can predict performance. 

\section{The Report}
The report will begin in \hyperref[ch:background]{Chapter 2} with a review of the literature on rowing training and performance and the use of machine learning in sports science. This will provide the sports science related knowledge necessary to understand the data collected and the approach used for analysis and model development. \hyperref[ch:data-collect-mng]{Chapter 3} will discuss the approach used to collect and manage participant data, including the ethical considerations taken. Next, \hyperref[ch:data-anyl-viz]{Chapter 4} will discuss the approach to developing a data analysis pipeline to provide particpants with feedback on their training data. This will cover steps taken to clean and standardise the incoming data, the general approach which guided analysis, and how visualisations were developed and deployed. \hyperref[ch:ml]{Chapter 5} will discuss the machine learning approach taken. Unfortunately, due to limitations in data collection and time, the model was not able to be fully developed. However, the data collected and the initial analysis performed will be discussed in this report. The machine learning goals were adapted to explore an approach to predict injuries in rowers, which will be discussed in \hyperref[ch:ml]{Chapter 5}. The report will conclude with a discussion of the results and potential future work in \hyperref[ch:discussion]{Chapter 6}.
