\chapter{\label{chap:discussion}Discussion}
This project provided many opportunities for learning and growth and, now, at its conclusion, there is an opportunity to reflect on the proposed goals of the project and its subsequent execution. Firstly, the initial target of developing a comprehensive machine-learning model for rowing performance was quite ambitious. This became clear in January when a data collection system was still being developed. Furthermore, the data collection goal of minimal user input directly contravened the development of a meaningful model. Nevertheless, the project was still successful in developing a data collection system, and a data cleaning pipeline, providing analysis and visualisations to a number of athletes who kindly volunteered their training data, and an exploration of a potential machine learning approach using a more complete data set which could be used to predict, and prevent, athlete burnout and injury. This foundational work could serve as a basis for future development of an effective model. 

This chapter will discuss and evaluate the methods used for data collection, management, cleaning, analysis and visualization, as well as discuss the potential for future work, particularly in developing a machine-learning model for predicting athlete performance. Finally, a protocol will be outlined for the effective implementation of a system to develop and deploy a performance model, based on the learnings gleaned throughout this project.

\section{Data collection}
\subsection{Evaluation}
The data collection system, based on the goals outlined at the start of the project, was successful in collecting data from a number of athletes, across four different data sources. Roughly 40 athletes were approached to provide data. Of that number, 12 athletes signed up, with around 8 consistently providing data and feedback on developments. This drop-off between the number of athletes registered for the platform and providing data was due to injuries or illness, preventing training. The data collection system was effective in providing a user-friendly platform for athletes to provide data with the help of the data provider's APIs. In total, since athlete recruitment was completed at the end of January, roughly 500 activities were ingested through the data collection pipeline. This number was lower than expected due to a recent, and poorly documented change by Strava in how developer apps are handled. This meant that roughly 6 weeks of training data for all but one user were not shared with the application, as a result, a number of user data sets were somewhat incomplete and there was not enough time to fully remedy this issue by backfilling athlete data. Analysis for these users was supported by data provided from Concept2, Polar, and, in the last 4 weeks, Garmin. The researcher's data set was preserved and used for developing the analysis and visualisation steps of the project.

\subsection{Discussion}
The data collection approach, while effective in minimising the effort of individual athletes, made the development of any kind of performance model difficult. The advertised automation of data collection made it easier to recruit athletes, both due to the minimal effort required to participate in the project, and because of the clear security by obfuscation for competitors. This did, unfortunately, result in a dataset which was not as complete as it could have been. The lack of heart rate data for many sessions, and the lack of detail when recording strength sessions, limited the analysis which could be performed. The lack of heart rate data, in particular, made it difficult to classify sessions as an endurance or interval session, and to calculate training impulse. For the purposes of research, a more involved group of participants is required. Athletes who are willing to take the time to provide feedback for sessions, and ensure their training logs are complete, make it easier to develop a model which can be used to predict performance, and perhaps facilitate the low-effort data collection and analysis approach this project strove for. 

There were also a number of different training plans being executed by participants, which made it difficult to develop a model which could be used to predict performance. A more consistent training plan, or a more detailed analysis of the training plans being executed, would be necessary to develop a model which could be used to predict performance. If it were possible to onboard an entire squad or have participants commit to completing a specific training session each week, there would be a clear metric for improvements, or decline, in fitness and therefore an anchor by which to compare an athlete's performance. This will be further discussed in the final section of this chapter, \autoref{sec:model-devel-prot}.

The data collection method developed throughout this project was successful in providing an easy, and secure, way for athletes to provide training data. The data collected was not as complete as it could have been, however, and this limited the analysis and model development which could be performed. This is an easy opportunity for future work, introducing session matching and feedback, a more detailed onboarding process, could help to ensure the data collected is more complete and useful for analysis. This also introduces the opportunity to develop daily monitoring surveys for athletes to complete to give a more complete picture of their training, recovery, and general well-being, something which can be used for more effective analysis.

\section{Data Analysis and Visualisation}
\section{Machine Learning}
\section{\label{sec:model-devel-prot}A proposed protocol for developing a performance model}