% \chapter*{Abstract}
\thispagestyle{plain}
\begin{center}
  \vspace{3cm}
  \textbf{\Huge\thesistitle}

  \begin{minipage}{12cm}
    \begin{center}
      \thesissubtitle
    \end{center}
  \end{minipage}

  \vspace{1cm}

  \authorname, \degree

  University of Dublin, Trinity College, 2024

  Supervisor: \supervisor

  \vspace{2cm}
  \textbf{Abstract}
\end{center}


Modelling human performance has been a challenging task for many years. The complexity of human physiology and the variety of factors that can influence performance make it difficult to develop accurate models. Machine learning, specifically neural networks, have proven particularly promising in applications to predict human performance. This project explores the approaches to data collection, analysis, and machine learning to predict rowing training outcomes and performances. The project aims to use machine learning techniques to predict performance using rowing data and explore the use of high-quality data to predict athlete injury. 

The project collected data from rowers using a web application developed as part of the project. This application was used to collect data on training sessions by allowing participants to connect data sources already used to track training with the project. The data collected was then analysed to develop features for a model, and visualisations were produced as feedback for participants. Unfortunately, the data collected was insufficient to develop a model to predict performance. Instead, the project explored the use of machine learning to predict injuries in runners, noting the approach and its potential application to appropriately collected rowing data.

Keywords: \keywords