% \chapter*{Abstract}
% \thispagestyle{empty}
\begin{center}
  \vspace{3cm}
  \textbf{\Huge\thesistitle}

  \begin{minipage}{12cm}
    \begin{center}
      \thesissubtitle
    \end{center}
  \end{minipage}

  \vspace{1cm}

  \authorname, \degree

  University of Dublin, Trinity College, 2024

  Supervisor: \supervisor

  \vspace{1.5cm}
  \textbf{Abstract}
\end{center}


Modelling human performance has been a challenging task for many years. The complexity of human physiology and the variety of factors that can influence performance make it difficult to develop accurate models. Machine learning, specifically the use of neural networks, has proven particularly promising in applications to predict human performance. This project explores the approaches to data collection, analysis, and machine learning to predict rowing training outcomes and performances. The project aims to use machine learning techniques to predict performance using rowing data and explore the use of high-quality data to predict athlete injury. 

A web application was developed to collect data from rowers as part of the project. This application was used to collect data on training sessions by allowing participants to connect data sources already used to track training with the project. The data collected was then analysed to develop features for a model, and visualisations were produced as feedback for participants. Unfortunately, the data collected was insufficient to develop a model to predict performance. Instead, the project explored the use of machine learning to predict injuries in runners, noting the approach and its potential application to rowing data.

Finally, this project suggests a protocol by which an improved data collection and analysis process can be implemented to train and produce effective machine learning models. These can be personalised and applied to understand individual athlete and crew training, injury, and illness behaviours. As a result of a more involved collection approach, deeper analysis can be completed to provide valuable feedback in the absence of individual performance models. This improved data collection and analysis will provide more potential features to train a variety of models, beginning with  the injury prediction and performance models suggested in this project. Each of these suggestions can be used to develop stronger athlete and crew performances by reducing injury and illness rates and adapting training to an individual's unique training response. Furthermore, this approach can deliver a more robust analysis of training sessions and athletes' qualitative and quantitative responses, leading to even greater success. This can also be adapted and applied to other sports, particularly endurance sports.

\vspace{1cm}
Keywords: \keywords 